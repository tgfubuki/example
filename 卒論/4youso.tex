%!TEX root = 0卒業論文.tex
\newpage

\section{\rm 製作する教材に組み込む要素}
本章では,教材を制作するにあたって,教材に組み込む要素をまとめた.

\subsection{小学校プログラミング教育での活用}
本研究で制作する教材は,総合的な学習の時間及び放課後のクラブ活動の時間で使用することを想定し, 一問一問を簡潔なプログラムなプログラムで作成し, プログラミングに対しての知識が不足している教員に対し,必要最低知識と授業内でも使用することができる例題を提供するものにする.

小学校プログラミング教育で目標としている要素である「知識・技能」,「思考力・判断力・表現力」,「学びに向かう力・人間性」より,教員が事前に学習し, 児童向けた授業で使用することを前提として,最適化させるために「知識」「思考力」の2点に絞った.

\subsection{思考力}
思考力の要素よりプログラミング的思考力を身につけるとある.このプログラミング的思考力をより具体化すると以下の論理的思考力となる.これらの論理的思考力より,基本的なシーケンス,ループ,分岐,真偽値の存在を認知するすることを目標とする.
\begin{itemize}
\item 計算や作業を手順に分けて順序立てる「順番処理」の考え方\\
\item 手順のまとまりを繰り返して実行する「ループ」の考え方\\
\item 条件によって作業を切り替える「分岐」の考え方\\
\item ものごとをYes/No の組み合わせで考える「真偽値」の考え方\\
\item ものごとの性質や手順のまとまりに名前をつける「抽象化」の考え方
\end{itemize}

\subsection{知識}
小学校教育ではこの「知識」要素より,身近な生活にコンピュータが活用されていることや,問題の解決に必要な手順があることに気づくことを目標にしている.児童に向けた授業内で使用するために簡易化させ,プログラミングに触れることでプログラミングという概念の存在を認知することを目標とする.

\subsection{プログラミングの面白さを知る}
プログラミングを学習するにあたってプログラミングの面白さを知ることは,学習意欲に大きく影響する.具体的にプログラミングの面白さの要素とは何なのか以下にまとめる.
\begin{itemize}
\item ものを作ること\\
\item 他人の役に立つこと\\
\item 複雑な仕組みを見ること\\
\item 新しいものを知ること\\
\item 扱いやすい道具を使うこと
\end{itemize}

\subsection{取り扱う教材}
上記より「\UTF{2460}知識」「\UTF{2461}思考力」を育む教材として,例題提供を行うドリル型の教材を制作する上で使用する教材に求める要素は,「プログラミングの面白さ」の5項目の中から「\UTF{2462}ものを作ること」「\UTF{2464}複雑な仕組みを見ること」「\UTF{2466}扱いやすい道具を使うこと」の3つであると考えた.これらを満たす教材として,3章で挙げた教材の中から,「Scratch」「Viscuit」「教育版レゴ」の3つを最初の候補とした.「教育版レゴ」は有料であり,さらに実際にレゴブロックを組み立てる必要があるため,今回制作する教材には不向きであると考え候補から除外した.また,「Viscuit」には既に公式HP内で指導者向けの資料を配布していることから, 本制作では無料かつブラウザで使用できる「Scratch」を対象にした例題を提供することとした.