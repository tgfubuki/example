%!TEX root = 0卒業論文.tex
\newpage

\section{\rm 小学校プログラミング必修化}
本章では,小学校プログラミング教育の必修化に至った背景や目標についてその説明をするとともに,教育指導要領の改定による小学校教員への影響を示す.

\subsection{必修化の背景}
小学校プログラミング教育の必修化の背景として,以下に文部科学省が公開した「小学校プログラミング教育の手引(第一版)」より抜粋する.

\begin{itemize}
 \item 誰にとっても,職業生活をはじめ,学校での学習や生涯学習,家庭生活や余暇生活など,あらゆる活動において,コンピュータなどの情報機器やサービスとそれによってもたらされる情報とを適切に選択・活用して問題を解決していくことが不可欠な社会が到来しつつあります.\\

 \item プログラミングによって,コンピュータに自分が求める動作をさせることができるとともに,コンピュータの仕組みの一端をうかがい知れることができるので,コンピュータが「魔法の箱」ではなくなり,より主体的に活用することにつながります.\\

 \item コンピュータを理解し上手に活用していく力を身につけることは,あらゆる活動においてコンピュータ等を活用することが求められるこれからの社会を生きていく子供達にとって,将来どのような職業に就くとしても,極めて重要なこととなっています.
 \end{itemize}

\subsection{プログラミング教育とは}
文部科学省の掲げるプログラミング教育とは,以下に文部科学省主催の有識者会議でまとめられたものから抜粋する.
\\
\\
``子供たちに,コンピュータに意図した処理を行うように指示することができるということを体験させながら,将来どのような職業に就くとしても,時代を超えて普遍的に求められる力としての「プログラミング的思考」を育成するもの.''\\

また,プログラミング的思考とは何か,以下に文部科学省主催の有識者会議でまとめられたものから抜粋する.
\\
\\
``自分が意図する一連の活動を実現するために,どのような動きの組み合わせが必要であり,1 つ1 つの
動きに対応した記号を,どのように組み合わせたらいいのか,記号の組み合わせをどのように改善して
いけば,より意図した活動に近づくのかということを論理的に考えていく力である.''


\subsection{プログラミング教育を通じて目指す育成すべき資質・能力}
プログラミング教育を通じて目指す育成すべき資質・能力について,以下に文部科学省主催の有識者会議でまとめられたものから抜粋する.\\
\\
【知識・技能】\\
身近な生活でコンピュータが活用されていることや,問題の解決には必要な手順があることに気づく
こと.\\
\\
【思考力・判断力・表現力】\\
発展の段階に即して,「プログラミング的思考」を育成すること.\\
\\
【学びに向かう力・人間性】\\
発達の段階に即して,コンピュータの動きを,よりよい人生や社会づくりに生かそうとする態度を涵
養すること.\\
\\

\subsection{小学校段階におけるプログラミング教育の実用例}
小学校で必修化されるプログラミング教育は,教科化ではない.そのため算数や国語のような教科ごとの時間は用意されない.そのため既存の教科にプログラミング要素を関連付けて教育することになる.以下に文部科学省主催の有識者会議でまとめられた実用例を抜粋する.\\\\

\begin{table}[htb]
    \caption{小学校段階におけるプログラミング教育の実用例}
  \begin{tabularx}{\linewidth}{|X|X|} \hline
    教科& 実用例 \\ \hline
    総合学習& 自分の暮らしとプログラミングとの関係を考え,そのよさに気づく学習 \\ \hline
    理科&電気製品にはプログラムが活用され条件に応じて動作していることに気づく学習\\ \hline
    算数 & 図の作成において,プログラミング的思考と数学的な思考の関係やよさに気づく学習\\ \hline
    音楽 & 創作用のICT ツールを活用しながら,音の長さや高さの組み合わせなど試行錯誤し,音楽をつくる学習\\ \hline
    図画工作 & 表現しているものを,プログラミングを通じて動かすことにより,新たな発送や構想を生み出す学習\\ \hline
   
  \end{tabularx}

  \end{table}
\newpage


\subsubsection{コンピュータを使わない授業の実践例}
小学校プログラミング必修化において実際に行われた実験例\cite{con}を以下にまとめる.\\
\\
\begin{table}[htb]
\begin{center}
\centering
    \caption{コンピュータを使わない授業の実践例}
  \begin{tabular}{|c|c|l|} \hline
    教科& 学年&実践例 \\ \hline
    国語& 3&文章の構成を順番処理(シーケンス)の考え方を用いて考える授業\\ \hline
    算数& 3&筆算の仕方を順番処理(シーケンス)の考え方を用いて考える授業\\ \hline
    理科& 3&身の回りの物を条件でグループ分けする授業\\ \hline
    音楽& 3&曲に合わせた三拍子のリズムをループなどを用いて構築する授業\\ \hline
    算数& 4&ベン図を用いた条件分けの授業\\ \hline
    学級活動& 4&仲間の個性を真偽クイズとして作成し,当て合うレクリエーション\\ \hline
    外国語活動& 6&マス目わけした地図を英語で指示をして最短経路で目的地まで進む\\ \hline
   
  \end{tabular}
  \label{tab:bamen1}
  \end{center}
\end{table}

\subsection{教育指導要領改定に伴う小学校教員への影響}
2020年の教育指導要領の改定は, 小学校におけるプログラミング授業の必修化だけではなく,外国語学習の必修化や理科・算数に統計を用いた授業が取り組まれるなど,様々な内容を含んでいる.

生活指導を主に行う小学校では, 全ての科目を担任の教員一人の授業で行っていることから, 教育指導要領の改定に伴い,外国語やプログラミングに対する知識や理系科目に対する深い理解も必要になる.しかし,教員たちがある程度の知識を有する外国語や理系科目に対し,プログラミングに対する知識がある教員は少ない.2018年度に教育ネットが行った東京都の小学校教員148名を対象とした「プログラミング教育に対する意識調査」[1]において, プログラミング授業を実践した経験についての項目では, 85%が「実施した経験はない」と回答した.経験者であっても「1回実践」9%がもっとも多く,「2回実践」は2%,「3回以上実践」は4%という結果であった.また, 授業を自身で実施することに不安があるかという質問では,「とても不安」が77%,「少し不安」が21%,「あまり不安はない」「まったく不安はない」は2%にとどまり,教員の98%が授業の実施に対して不安を感じているというデータもある.

この知識不足による不安を解消するために教員は自力で学習をすることや研修に参加することが求められる.しかし, 教員は既存の教育に加えてこれを行う必要があるため,授業を行うにあたっての事前学習時間の増加による業務負担の増加が教育指導要領改定による影響として挙げられる.

\subsection{考察}
小学校プログラミング教育で活用することを目指すため有識者会議でまとめられた要素を取り組む必要がある.既に行われているプログラミング教育の実践例,実用例より順番処理やループといった論理的思考力を高めるといった目的をうまく既存の教材に取り入れ,実践していることがわかる.

しかし,教材の使用方法などについては個別で学習を進めていかなくてはならなず, ループや条件分岐といった処理を活用する際はその挙動についても理解しておくことが必要不可欠である.また,プログラミングの授業を行うために教員のプログラミングに対する知識・経験不足から来る不安を解消する必要もある.

そのため, 国語,算数といった既存科目でプログラミングを行う以前の学習として,総合学習の時間や放課後のクラブ活動の時間などを使い, 教員と児童に対して一通りの操作やプログラムの挙動を教えるための教材を制作する必要がある.それに加え,プログラムを作り上げた際に,どのような挙動を行うかを実際に見せながらどのようなプログラムを作成すればいいのかを考えさせることで,文部科学省の提唱する「プログラミング的思考力」を育成することを目的とした教材を制作することが良いと考える.




%【参考文献(産経WEST)など】